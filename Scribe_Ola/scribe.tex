%Author - Akshay Vijayvergia
% $DA-IICT ID$ 201401061.


\documentclass[11pt]{article}

\oddsidemargin  -0.0in
\textwidth      6.5in
\headheight     0.0in
\topmargin      -0.25in
\textheight     9.0in
\parindent      0.0in
\parskip        0.05in

\usepackage{epsfig}
\usepackage{listings}

%used to include animation
\usepackage{animate}
\usepackage{animfp}

\begin{document}

% ---------------------------------------------------------------------------

\begin{center}
{\large SC462 Elements of Synthetic Biology: Life 2.0} \\
Prof. Manish K Gupta \\
DA-IICT\\
\bigskip
\hspace{0.25in} \\
{\Large\bf Cello: genetic circuit design automation} \\
Akshay Vijayvergia
\end{center}

% ---------------------------------------------------------------------------
\bigskip
\bigskip
\bigskip
\bigskip
\section*{Abstract}
DNA is the single most important part of all the organisms present on the planet. It consists of the genetic instructions used in the functioning, growth, and development of all known living organisms. Due to this, to study biological architecture, we need to perform computations in living cells by DNA-encoded circuits which indeed process sensory information and control all the biological functions. But due to the various complexions like time and design, these computations become close to impossible. Thus to tackle this problem, we study Cello\cite{ncbi}, which serves as a design environment where the user gives a Verilog code as input that is automatically transformed into a DNA sequence providing a genetic circuit design. 

%%%%%%%%%%%%%%%%%%%%%

\bigskip
\section*{Introduction}

You could say "I want a cell that can detect certain elements in water and provide different color output signals. Also, the output signals generated should correlate to the concentration of the elements specified detected." .These are the types of problems which could be efficiently simulated by Cello. Cello is a web based open source tool that is used to build genetic circuit diagrams. It's a joint initiative between Boston University and MIT, initiated by the CIDAR lab at Boston.\cite{Reddit}

Its input consists of a Verilog script. This script is then parsed into a truth table which is further synthesized to generate a genetic circuit diagram with the genetically available gate types in the library. For simulating the circuit, a predicted score guides a breadth-first search or a Monte Carlo simulated annealing search. Among all the predicted scores, the one with the highest score is selected. The one which is selected could then be implemented using the various genetic layouts. For finalizing among one or more DNA sequences for the designated circuit, the Eugene language is used which further processes upon the rule based combinatorial design.

To briefly analyze Cello, firstly, we start with the actual/internal working of cello and begin to visualize how the Verilog input is converted to a desired genomic circuit. Later we look into an example and find out the various functionalities provided by the web app.
%%%%%%%%%%%%%%%%%%%%%

\section*{Working}
\bigskip
\subsection*{VERILOG SPPECIFICATION}
When opening the website http://www.cellocad.org/verilog.html, we are greeted with a text editor which enables us to input the Verilog script. Cello accepts three forms of Verilog scripts namely case statements, assign statements and structural statements. The Verilog programs usually start with module keyword, followed by the name taking a list of output and input wire names as its arguments. We now focus on all the three specified Verilog types.

\begin{enumerate}
\item Case: 
This style is usually preferred when we directly want to specify the truth table for our circuit as shown in figure~\ref{Case statements exmaple}. \bigskip
\begin{figure}[ht!]
\centering
\includegraphics[width=4cm,height=12cm,keepaspectratio]{case_1.png}
\includegraphics[width=11cm,height=14cm,keepaspectratio]{case_2.png}
\label{Case statements exmaple}
\end{figure}

\item Assign: 
This style is usually preferred when we want to specify circuit using boolean operators. For detailing the order of operations, parentheses are used. One thing should be noted that all the internal wires must be defined before use as shown in figure~\ref{Assign statements exmaple}.\bigskip
\begin{figure}[ht!]
\centering
\includegraphics[width=11cm,height=14cm,keepaspectratio]{assign.png}
\label{Assign statements exmaple}
\end{figure}

\item Structural: 
This style is usually preferred when we want a gate-level wiring diagram to be specified. Again all the internal wires need to be defined before use as shown in figure~\ref{Structural statements exmaple}.  
\begin{figure}[ht!]
\centering
\includegraphics[width=11cm,height=14cm,keepaspectratio]{structural.png}
\label{Structural statements exmaple}
\end{figure}

Here, within, the parenthesis, the first argument is the output wire and all the other trailing arguments are the inputs. The symbol at the beginning of each line is an operator.\\[\baselineskip]     
\bigskip
\bigskip
\bigskip
\bigskip
\end{enumerate}
By providing the Verilog input in any of the above three designs, we then move onto the logical synthesis step.

\subsection*{LOGIC SYNTHESIS}
The Verilog script is then parsed to generate a truth table. The truth table is then converted to the final wiring diagram in the following steps.
\begin{enumerate}
\item AND-Inverter Graph:
The table is converted to an AND-Inverted graph consisting of 2-input AND gates and NOT gates.
\item NOR-Inverter Graph:
The above is then converted to a NOR-Inverter Graph using DeMorgan's rule
\item Subcircuit substitutions:
The above is then simplified and further substituted with smaller but functionally equivalent logic diagrams in order to reduce the number of gates. In this step, other gate types can also be included.\cite{Cidarlab}
\end{enumerate}
The above process is better described in the figure shown below in figure~\ref{Logic Synthesis}.
\begin{figure}[ht!]
\centering
\includegraphics[width=18cm,height=18cm,keepaspectratio]{logic.png}
\label{Logic Synthesis}
\end{figure}
\\[\baselineskip]    
After creating the simplified wiring diagram, we then assign gates to the diagram from the Cello library as per specification detailed in the next subsection.

\subsection*{GATE ASSIGNMENT}
Some basic gates in the gate library are shown below in figure~\ref{Gate Library}. If we want to add other gates to our library, we can add those using the UCF
\begin{figure}[ht!]
\centering
\includegraphics[width=16cm,height=16cm,keepaspectratio]{gate.png}
\label{Gate Library}
\end{figure}
\\[\baselineskip]    
All the gates shown above have an experimentally determined response function that relates one or more input values to an output value which is specified in standardized units (RPU = Relative Promoter Units). These response functions are then fitted to a Hill function which takes ymax, ymin, K,n as its 4 arguments.
The response functions of the above figured gates are shown below in figure ~\ref{Response functions for library gates}: 
\begin{figure}[ht!]
\centering
\includegraphics[width=16cm,height=17cm,keepaspectratio]{gate_response.png}
\label{Response functions for library gates}
\end{figure}

But still, the main question remains as to how do we assign gates to all the elements in the circuit. This assignment is done using the following assignment algorithms.
\begin{enumerate}
\item Breadth-first search: 
This algorithm guarantees to select a genetic circuit with  a global max score among all other assignments. It starts with the gates which are closest to the inputs. The algorithm then goes one gate at a time where signal mismatches are faulty. For each acceptance in the traversal then next breadth first order gate is exhaustively assigned following the same procedure again until all the gates have been visited. After the completion, the circuit with the highest assignment score is selected.  \cite{Cidarlab}
\item Hill climbing: 
This algorithm starts with a random assignment, and then swapping the assignment of 2 gates. Over here, gate 1 is randomly selected from the circuit to process and gate 2 is selected from the circuit or the gates which are left unused in the gate library for cello. The above specified swap is accepted if the initially initialized circuit score increases. After running this extensively on the whole circuit, the circuit with the highest assignment score is selected.\cite{Cidarlab}
\item Simulated annealing: 
This is same as hill climbing except from the difference that the swaps that decrease circuit score with a probability that recombines/anneals over time are selected.\cite{Cidarlab}
\end{enumerate}
Now for a given assignment, we can have a lot of combination of the arranging the genetic parts into physical diagrams. This problem is solved and assessed using the Eugene specification. The final DNA sequences that result from the Eugene variants is then placed into the genomic locations specified in the earlier describes UCF.   

\section*{Example}
Now let us take a verilog script and run it on cello webapp to find out the various logs and outputs provided by the software.\cite{Cellocad}
\begin{figure}[ht!]
\centering
\includegraphics[width=10cm,height=11cm,keepaspectratio]{ex_input.png}
\label{Sample case input for AND gate}
\end{figure}
\\[\baselineskip]    
On running the above case script in figure~\ref{Sample case input for AND gate} for AND gate and selecting the inputs as pTac and pTet along with the output as RPF, we firstly get the following gate assignment after its logical synthesis as shown below in figure~\ref{Logic synthesis for AND gate}:  
\begin{figure}[ht!]
\centering
\includegraphics[width=16cm,height=16cm,keepaspectratio]{ex_synthesis.png}
\label{Logic synthesis for AND gate}
\end{figure}
\\[\baselineskip]   
\\[\baselineskip]   
\\[\baselineskip]   
\\[\baselineskip]   
The Response function of the above gate assignment circuit is shown below in figure~\ref{Response function for gate assignment}:
\\[\baselineskip]   
\begin{figure}[ht!]
\centering
\includegraphics[width=16cm,height=16cm,keepaspectratio]{ex_response.png}
\label{Response function for gate assignment}
\end{figure}
\\[\baselineskip]   
As stated before, in the gate assignment subsection, all the gates have a  predefined output in RPUs(Relative Promoter Units). For the circuit, we see the predicted RPUs of the gates as shown below in figure ~\ref{Predicted RPU of gate}.
\begin{figure}[ht!]
\centering
\includegraphics[width=16cm,height=16cm,keepaspectratio]{ex_rpu.png}
\label{Predicted RPU of gate}
\end{figure}
\\[\baselineskip]   \\[\baselineskip]   \\[\baselineskip]   \\[\baselineskip]   
On providing the assignment algorithms we get the predicted gate output as shown below in figure~\ref{Predicted RPU of the gate}:
\begin{figure}[ht!]
\centering
\includegraphics[width=11cm,height=11cm,keepaspectratio]{ex_rpuOutput.png}
\label{Predicted RPU of the gate}
\end{figure}
\\[\baselineskip]   
The final circuit diagram created by the software is shown below in figure~\ref{Circuit diagram} after processing through the Eugene rules.
\begin{figure}[ht!]
\centering
\includegraphics[width=13cm,height=14cm,keepaspectratio]{ex_output.png}
\label{Circuit diagram}
\end{figure}
\\[\baselineskip] 
In all the steps shown above, log files are also generated which further provide a detailed understanding of all the above steps as simulated by the Cello software.
In the end we are greeted with sbol and plasmid log files which are indeed the results that we needed. Some snapshots of the logs files are attached below for reference
 \\[\baselineskip]   
 \\[\baselineskip]   
\begin{figure}[ht!]
\centering
\includegraphics[width=8cm,height=20cm,keepaspectratio]{fout_1.png}
\includegraphics[width=8cm,height=20cm,keepaspectratio]{fout_2.png}
\label{sbol and plasmid log snapshots}
\end{figure}


\section*{Discussions}
The environment surrounding a cells allows it to respond to make decisions, coordinate tasks and build structures. Coordinating all these processes are computational operations performed by dense networks of proteins that control the timing of gene expression. Through research\cite{journal}
 it has been shown that cells ca programmed using synthetic circuits composed of regulators organised to generate the desired operation. 
\\[\baselineskip]  One such enthusiastic group applied Cello to design 60 circuits of Escherichia coli. Here, the circuit function was specified using Verilog script and then further converted into a DNA sequence. These DNA sequences required 8,80,000 base pairs of DNA assembly without any additional tuning. Out of these, 45 circuits performed correctly in every given output state. Across all of the implementations 92 percent of the 412 output states functioned as predicted. Some detailed visuals of the experiment as shown below:
\begin{figure}[ht!]
\centering
\includegraphics[width=8cm,height=10cm,keepaspectratio]{discussion1.jpg}
\label{d_input}
\end{figure}
\begin{figure}[ht!]
\centering
\includegraphics[width=18cm,height=13cm,keepaspectratio]{discussion2.jpg}
\label{d_output}
\end{figure}

% ========================================================================


\bibliography{biblio_cello}
\bibliographystyle{plain} 
\end{document}